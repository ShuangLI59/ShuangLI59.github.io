%%--------------------------------------------------------------------------
\chapter*{摘要}
\addcontentsline{toc}{chapter}{Abstract in Chinese}
\markboth{\uppercase{Abstract}}{\uppercase{Abstract}}
%%--------------------------------------------------------------------------
人物檢索通過圖像匹配在數據庫中找到目標人物。人物檢索廣泛應用在實際場景中,例如人物跟蹤,室內安防,以及自動駕駛。本文針對人物檢索的三個問題進行了詳細的分析,包括基於視頻的人物檢索,基於自然語言描述的人物檢索,以及基於個體的文字圖像匹配。

%在論文的第一部分,我們提出了一個深度神經網絡,該網絡基於全景圖像聯合優化行人檢測和行人再識別兩個子任務。傳統的Softmax損失函數在解決大規模的驗證問題時會引入偏差,我們因此設計了一個新的損失函數去保證網絡能夠高效的訓練。
在論文的第一部分,我們提出了一個多樣性約束的時空注意力模型去解決視頻人物檢索。由於行人經常有部分區域被遮擋,時空注意力模型並沒有平均每一幀圖像的特徵作為視頻特徵,它學習去檢測多個顯著區域並分別提取每個區域的特徵從而避免特徵被遮擋的區域污染。一個多樣性約束項被提出來保證多個空間注意力模型檢測到人體的不同部位。

基於視頻的人物檢索並不能解決全部實際問題,例如當只有語言描述可以利用時,基於視頻檢索的方法將失效。論文的第二部分提出了基於自然語言描述的人物檢索。我們建立了一個描述人物的數據庫並且設計了一個深度神經網絡去捕捉文字圖像關係。該神經網絡由一個視覺子網絡和一個語言子網絡組成。其中視覺子網絡編碼特定的人物特徵和外觀模式。語言子網絡是一個卷積神經網絡,它通過預測每個圖像區域針對當前輸入文字的重要程度來計算文字圖像的相關性。

基於自然語言描述的人物檢索面臨如何在單一網絡中平衡準確性和高效性的問題。在論文的第三部分,我們提出了一個雙階段的方法去解決這個問題。在第一階段,我們通過交叉熵損失函數過濾掉簡單的錯誤文字圖像對。第二階段利用聯合注意力模型針對剩下的複雜文字圖像對進行預測。
這個方法可以用來解決基本的文字圖像匹配類問題。
%該行人檢索模型結果對自然語言描述的結構更加魯邦並且取得了當前最好的結果。